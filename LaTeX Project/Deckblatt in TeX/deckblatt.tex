% das Papierformat zuerst
\documentclass[a4paper, 11pt]{article}

% deutsche Silbentrennung
\usepackage[ngerman]{babel}

% wegen deutschen Umlauten
\usepackage[ansinew]{inputenc}

% hier beginnt das Dokument
\begin{document}


\thispagestyle{empty}
\begin{center}
\Large{Hochschule Ulm}\\
\end{center}

\begin{verbatim}





\end{verbatim}
\begin{center}
\textbf{\LARGE{Master-Thesis}}
\end{center}
\begin{verbatim}


\end{verbatim}
\begin{center}
\textbf{im Studiengang\\
	Systems Engineering and Management\\
	Electrical Engineering}
\end{center}
\begin{verbatim}














\end{verbatim}

\begin{flushleft}
\begin{tabular}{lll}
\textbf{Thema:} & & Entwicklung und Realisierung einer modularen\\
& & Klimaregelung auf Basis eines Klimamodells (Arbeitstitel)\\
& & \\
& & \\
& & \\
\textbf{eingereicht von:} & & Ren� Trampenau\\
& & \\
& & \\
\textbf{eingereicht am:} & & 15. September 2014\\
& & \\
& & \\
\textbf{Betreuer:} & & Herr Prof. Dr.-Ing. W. Schroer\\
 & & Herr Prof. Dr. rer.nat. D. Bank
\end{tabular}
\end{flushleft}

% das ist wohl jetzt das Ende des Dokumentes
\end{document}
