\documentclass[11pt,a4paper]{article}
\usepackage[latin1]{inputenc}
\usepackage{amsmath}
\usepackage{amsfonts}
\usepackage{amssymb}
\usepackage{makeidx}
\usepackage{graphicx}
\usepackage{listings} %f�r Listings (Quellcode)
\usepackage{xcolor} %f�r farbigen Quellcode (Codehighlighting)
\usepackage{hyperref} %F�r Hyperlinks


\newcommand{\zb}{z.\,B. } % \zb wird nun mit z. B. ersetzt

\author{Ren� Trampenau}
\numberwithin{equation}{section}

\begin{document}

\section{Matrizen}


\subsection{Ohne Klammern}
\begin{equation}
	\begin{matrix} a_1 & a_2 & a_3\\ b_1 & b_2 & b_3 \\ c_1 & c_2 & c_3 \end{matrix}
\end{equation}

\subsection{Runde Klammern}
\begin{equation}
	\begin{pmatrix} a_1 & a_2 & a_3\\ b_1 & b_2 & b_3 \\ c_1 & c_2 & c_3 \end{pmatrix}
\end{equation}

\subsection{Eckige Klammern}
\begin{equation}
	\begin{bmatrix} a_1 & a_2 & a_3\\ b_1 & b_2 & b_3 \\ c_1 & c_2 & c_3 \end{bmatrix}
\end{equation}

\subsection{Geschweifte Klammern}
\begin{equation}
	\begin{Bmatrix} a_1 & a_2 & a_3\\ b_1 & b_2 & b_3 \\ c_1 & c_2 & c_3 \end{Bmatrix}
\end{equation}

\subsection{Betragsstriche}
\begin{equation}
	\begin{vmatrix} a_1 & a_2 & a_3\\ b_1 & b_2 & b_3 \\ c_1 & c_2 & c_3 \end{vmatrix}
\end{equation}

\subsection{Doppelte Betragsstriche (Norm)}
\begin{equation}
	\begin{Vmatrix} a_1 & a_2 & a_3\\ b_1 & b_2 & b_3 \\ c_1 & c_2 & c_3 \end{Vmatrix}
\end{equation}

\newpage

\section{Matlab Quellcode}

\subsection{Was wird ben�tigt}
F�r Listings (Quellcode-Fenster) ben�tigt man lediglich das Package:\\
\begin{lstlisting}
\usepackage{listings}
\end{lstlisting}
M�chte man den Quellcode einf�rben, ben�tigt man zus�tzlich:\\
\begin{lstlisting}
\usepackage{color}
\end{lstlisting}


\subsection{Direkt in Latex}
\begin{lstlisting}
a = 1;
b = 2;
c = a + b;
\end{lstlisting}

\lstset{language = matlab, breaklines=true}
\subsection{Aus einer Datei}
\lstinputlisting{optimizer_v01.m}


\subsection{Quellcode im Flie�text}
Quellcode kann auch mitten im Flie�text eingebettet werden, wie \zb  hier: \lstinline|print "hello world"|.

\subsection{Mit Zeilennummern}
%Abstand zwischen Code und Zeilennummer �ndern: numbersep=5pt 
\lstset{numbers=left, numberstyle=\tiny, numbersep=5pt} 
\lstset{language=Matlab}
\begin{lstlisting}
a = 1;
b = 2;
c = a + b;
\end{lstlisting}

\subsection{Mit Kasten}
\lstset{frame = single}
\begin{lstlisting}
a = 1;
b = 2.5;
c = a + b;
ceil(c);
\end{lstlisting}

\subsection{Mit Farben}

\definecolor{mygreen}{rgb}{0,0.5,0}
\definecolor{mygray}{rgb}{0.5,0.5,0.5}
\definecolor{mymauve}{rgb}{0.58,0,0.82}
\definecolor{myyellow}{rgb}{1,1,0.88}
\definecolor{colString}{RGB}{160,32,240}

\lstset{
commentstyle=\color{mygreen},    % comment style
keywordstyle=\color{blue},       % keyword style
numberstyle=\tiny\color{mygray}, % the style that is used for the line-numbers
stringstyle={\color{colString}},%
backgroundcolor=\color{myyellow},   % choose the background color; you must add 
}
\lstset{language=Matlab, frame = single}
\begin{lstlisting}
a = 1;
b = 2.5;
c = a + b;
ceil(c)
\end{lstlisting}

\subsection{Mit anderen Zeilennummern}

\lstset{firstnumber=23}
\begin{lstlisting}
a = 1;
b = 2.5;
c = a + b;
ceil(c)
\end{lstlisting}

\subsection{Kleinere Schriftart}
\lstset{basicstyle=\ttfamily\scriptsize}
\begin{lstlisting}
a = 1;
b = 2.5;
c = a + b;
ceil(c)
\end{lstlisting}

\subsection{Nochmal alles zusammen}
\lstset{firstnumber=59}
\lstinputlisting{optimizer_v01.m}

\subsection{Und nochmal Quellcode im Flie�text}
Jetzt sieht auch der Quellcode im Flie�text anders aus, \zb  hier: \lstinline|print "hello world"|.

\subsection{Quellen}
\begin{list}{-}{}
	\item Allgemein: \url{http://wiki.infostudium.de/wiki/Quelltext_in_LaTeX}
	\item Andere Zeilennummern: \url{http://en.wikibooks.org/wiki/LaTeX/Source_Code_Listings}
\end{list}


\section{Grafiken}
\subsection{Simulink}





\end{document}