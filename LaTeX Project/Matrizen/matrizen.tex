\documentclass[11pt,a4paper]{article}
\usepackage[latin1]{inputenc}
\usepackage{amsmath}
\usepackage{amsfonts}
\usepackage{amssymb}
\usepackage{makeidx}
\usepackage{graphicx}
\usepackage{listings} %f�r Listings (Quellcode)
\usepackage{color} %f�r farbigen Quellcode (Codehighlighting)
\author{Ren� Trampenau}
\numberwithin{equation}{section}

\begin{document}

\section{Matrizen}


\subsection{Ohne Klammern}
\begin{equation}
	\begin{matrix} a_1 & a_2 & a_3\\ b_1 & b_2 & b_3 \\ c_1 & c_2 & c_3 \end{matrix}
\end{equation}

\subsection{Runde Klammern}
\begin{equation}
	\begin{pmatrix} a_1 & a_2 & a_3\\ b_1 & b_2 & b_3 \\ c_1 & c_2 & c_3 \end{pmatrix}
\end{equation}

\subsection{Eckige Klammern}
\begin{equation}
	\begin{bmatrix} a_1 & a_2 & a_3\\ b_1 & b_2 & b_3 \\ c_1 & c_2 & c_3 \end{bmatrix}
\end{equation}

\subsection{Geschweifte Klammern}
\begin{equation}
	\begin{Bmatrix} a_1 & a_2 & a_3\\ b_1 & b_2 & b_3 \\ c_1 & c_2 & c_3 \end{Bmatrix}
\end{equation}

\subsection{Betragsstriche}
\begin{equation}
	\begin{vmatrix} a_1 & a_2 & a_3\\ b_1 & b_2 & b_3 \\ c_1 & c_2 & c_3 \end{vmatrix}
\end{equation}

\subsection{Doppelte Betragsstriche (Norm)}
\begin{equation}
	\begin{Vmatrix} a_1 & a_2 & a_3\\ b_1 & b_2 & b_3 \\ c_1 & c_2 & c_3 \end{Vmatrix}
\end{equation}

\end{document}