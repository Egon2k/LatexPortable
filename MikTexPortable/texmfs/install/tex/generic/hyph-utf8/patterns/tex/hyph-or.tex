% These patterns originate from
%    https://github.com/santhoshtr/hyphenation/)
% and have been adapted for hyph-utf8 (for use in TeX).
%
%  Hyphenation for Oriya
%  Copyright (C) 2016 Santhosh Thottingal (santhosh dot thottingal at gmail dot com)
%
%  Permission is hereby granted, free of charge, to any person obtaining
%  a copy of this software and associated documentation files (the
%  "Software"), to deal in the Software without restriction, including
%  without limitation the rights to use, copy, modify, merge, publish,
%  distribute, sublicense, and/or sell copies of the Software, and to
%  permit persons to whom the Software is furnished to do so, subject to
%  the following conditions:
%
%  The above copyright notice and this permission notice shall be
%  included in all copies or substantial portions of the Software.
%
%  THE SOFTWARE IS PROVIDED "AS IS", WITHOUT WARRANTY OF ANY KIND,
%  EXPRESS OR IMPLIED, INCLUDING BUT NOT LIMITED TO THE WARRANTIES OF
%  MERCHANTABILITY, FITNESS FOR A PARTICULAR PURPOSE AND
%  NONINFRINGEMENT. IN NO EVENT SHALL THE AUTHORS OR COPYRIGHT HOLDERS BE
%  LIABLE FOR ANY CLAIM, DAMAGES OR OTHER LIABILITY, WHETHER IN AN ACTION
%  OF CONTRACT, TORT OR OTHERWISE, ARISING FROM, OUT OF OR IN CONNECTION
%  WITH THE SOFTWARE OR THE USE OR OTHER DEALINGS IN THE SOFTWARE.
%
\patterns{
% GENERAL RULE
% Do not break either side of ZERO-WIDTH JOINER  (U+200D)
2‍2
% Break on both sides of ZERO-WIDTH NON JOINER  (U+200C)
1‌1
% Break before or after any independent vowel.
ଅ1
ଆ1
ଇ1
ଈ1
ଉ1
ଊ1
ଋ1
ୠ1
ଌ1
ୡ1
ଏ1
ଐ1
ଓ1
ଔ1
% Break after any dependent vowel, but not before.
ା1
ି1
ୀ1
ୁ1
ୂ1
ୃ1
େ1
ୈ1
ୋ1
ୌ1
% Break before or after any consonant.
1କ
1ଖ
1ଗ
1ଘ
1ଙ
1ଚ
1ଛ
1ଜ
1ଝ
1ଞ
1ଟ
1ଠ
1ଡ
1ଢ
1ଣ
1ତ
1ଥ
1ଦ
1ଧ
1ନ
1ପ
1ଫ
1ବ
1ଭ
1ମ
1ଯ
1ର
1ଲ
1ଳ
1ଵ
1ଶ
1ଷ
1ସ
1ହ
% Do not break before anusvara, visarga and length mark.
2ଂ1
2ଃ1
2ୗ1
2ଁ1
% Do not break either side of virama (may be within conjunct).
2୍2
}
